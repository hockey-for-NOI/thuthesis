\chapter{引言}
\label{cha:intro}

CESM ~\cite{CESMwebsite}模型是一类应用较为广泛的自然气候变化模型,它包括大气模型,陆地模型,海洋模型等子模型。单个模型专注考虑某一个区域的物理过程,因此在与其他区域的接触面需要提供记录数据或联合其他模型互相提供数据才能尽可能模拟出真实过程。因此一套完整的模型运行流程由多个模型轮流运行互相提供数据组成。

耦合技术是一种专注于多个模型中数据交互的控制技术,其将每个模型封装成模式分量,然后由耦合器控制整体的运行时序、数据交互、存储恢复、统计分析等。其优点在于顶层耦合器封装了模式分量的运行细节,只将所需的数据暴露给分析部分,将底层实现与顶层分析尽可能隔离开来,从而能够平滑地替换不同模式分量,能够以较少的工作量进行不同模型组合的实验,或进行等位模式分量的对照实验。

由于耦合技术的重要性,国家重点研发计划大规模多模式多过程地球系统模式耦合平台研发项目提出了在架构上支持灵活快捷按需耦合的耦合描述方法研究子课题,现在已经设计并编写了一种可配置、基于代码生成的耦合代码生成框架,并已经通过了海气耦合实验验证。

本文基于该耦合代码生成框架对 CESM 的现存模式分量进行适配,在封装 CESM 代码的同时根据具体需求对耦合代码生成框架进行一定的功能补充和优化,最终使得 CESM 模型实例能够在我们自己的耦合代码生成平台上复现,完成对自主研发耦合平台的实现与验证。