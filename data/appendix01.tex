\chapter{开题报告暨相关外文资料调研阅读报告}

\section{项目背景}

CESM模型是一类应用较为广泛的自然气候变化模型,它包括大气模型,陆地模型,海洋模型等子模型。每个模型主要内容为求解偏微分方程,而不同子模型间互相提供边界值。近年来由于各种新模型的不断衍生,存在将不同子模型进行对比的需求。而原本CESM的代码内部依赖逻辑较为复杂,使得将子模型切分并重组较为困难。为了解决这个问题,耦合器定义了一套标准接口,并将各个子模型切分开来运用这一套接口独立运行,再通过耦合器内部进行边界值通信,最终得以复现原本CESM模型的运行,从而使得更换子模型进行对比变得较为简单。目前学长即将完成耦合器的第一版编写,我的当前任务是将第一个大气模型CAM配置运行成功,并将其拆解依赖后重新封装为耦合器标准接口库。为此,我阅读了CESM论文\cite{OPENCESM} ,CAM论文 \cite{OPENCAM} ,以及与我之后的研究方向:耦合器通信相关的MCT论文 \cite{OPENMCT1} \cite{OPENMCT2}。

\section{CESM论文阅读报告}

论文:

THE COMMUNITY EARTH SYSTEM MODEL (CESM) LARGE ENSEMBLE PROJECT A Community Resource for Studying Climate Change in the Presence of Internal Climate Variability by J. E. Kay, C. Deser, A. Phillips, A. Mai, C. Hannay, G. Strand, J. M. Arblaster, S. C. Bates, G. Danabasoglu, J. Edwards, M. Holland, P. Kushner, J.-F. Lamarque, D. Lawrence, K. Lindsay, A. Middleton, E. Munoz, R. Neale, K. Oleson, L. Polvani, and M. Vertenstein

概要:

CESM是一个研究非受迫气候变化的模型,其主要目的是通过内部气候变化研究对气候预测提供参考。CESM包含大气子模型,海洋子模型,陆地子模型,极地子模型等,并可以求解其耦合作用。其设计动机是各个子模型通常只能考虑到单个方面因素的作用,因而受到其他方面因素的影响较大且难以控制模型误差,因此考虑把多个模型联合求解。文章主要作为CESM模型的文档和简要实验结果。

不同模型有着不同的迭代周期,这是因为洋流在长达几千年的时间跨度上的变化几乎不会影响模型结果,但大气,陆地和极地收到外部环境影响较为严重,需要短至每数周长至一年就要更新外部观测数据。

模型共进行了30个实验,除去2个实验作为对照组之外,剩下28个实验使用完全相同的外部数据和带有微小误差的初始值数据,来通过统计对模型的误差和置信度进行估计。从结果上看,这些模型对全球变暖程度的预估在合理范围内,从而证明了内部气候变化对气候预测能够起到一定的作用。

CESM模型的分辨率通常较低,这一方面是因为计算复杂度受限,另一方面则是由于模型精度导致。在局部地区实验中,只有微小误差的原始数据在不同模型中得到了截然不同的结果(气温相差接近1万度)。这一定程度上反映了气候变化中的蝴蝶效应,同时也说明了内部气候变化预测的局限性。为了解决这一问题,一些针对特定地区建立的高分辨率子模型逐渐被开发出来。它们通过毗邻的低分辨率模型作为初值进行迭代求解,并通过其自身的高分辨率尽可能缩小误差。预测时间被缩短以减少蝴蝶效应的影响,同时区域面积的缩小也使得提高分辨率的模型依然可以在可接受的时间内得出结果。

文章对比了CESM和CMIP5模型。CMIP5的不同模型采用了不同的公式,而CESM模型控制外部参数不变,只有初值有微量的误差,因而在控制变量方面做的更好,可以得出更加有说服力的结论。


气候预测的一个重要目的是预测极端天气。而由于分辨率低下的原因,通常的模型对极端天气的适应度并不好。因此,CESM将极端天气作为外部因素输入到模型中,使得模型可以更加关注和拟合通常情况的气候变化。

总的来说,CESM对建立新的子模型提供了大量的外部参数处理和误差分析工作,使得一个具有局限的模型可以接入CESM而得出较为可靠的结果。

CESM的设计初衷和本文现在试图进行的实验相似度很高。然而本质的不同是,CESM注重于解决模型在理论计算上的局限性,编写者大多为研究气候、地理的科学家,而本文将要实现的耦合器注重于降低接入新模型的工作量,即将这些科学家写的代码转化为成体系的库。

\section{CAM论文阅读报告}

论文:

Exploratory High-Resolution Climate Simulations using the Community Atmosphere Model (CAM) JULIO T. BACMEISTER Atmospheric Modeling and Prediction Section, National Center for Atmospheric Research,* Boulder, Colorado MICHAEL F. WEHNER Lawrence Berkeley National Laboratory, Berkeley, California RICHARD B. NEALE, ANDREW GETTELMAN, CECILE HANNAY, PETER H. LAURITZEN, JULIE M. CARON, AND JOHN E. TRUESDALE Atmospheric Modeling and Prediction Section, National Center for Atmospheric Research,* Boulder, Colorado


概要:

CAM是CESM模型中被广泛采用的大气子模型。文章主要对比不同分辨率下模型的结果精度,以及提出了一种针对热带气旋的分布模型和复现热带气旋的一些技巧。文章讲述了CAM模型的参数调节,实验步骤,输入数据,以及针对高分辨率的一些调整。文章还叙述了一些特定的气候现象以及捕捉它们所需的分辨率,并针对这些气候现象建立了基础模型。最后,文章对一些特定时间和特定区域的实验进行了说明。

由于具体公式只影响模型本身,而对于耦合器来说模型计算部分可以看作黑箱,本文只需要关心模型的外部输入,参数设定和边界值传递,因此公式部分暂且略过。

从实验结果来看,高分辨率模型的降水概率提高,这是因为高分辨率模型捕捉到了更多的降雨事件,但从总体来看平均降雨量并没有发生显著变化。这意味着统计降雨量并不需要过高分辨率的模型,但高分辨率模型可以更加精确的预测某个特定位置是否降雨。这说明了不同目标对不同模型的需求。

针对不同季节的模型测试中发现,不同季节可能会产生不同种类的天气现象,而并没有一个模型能同时很好地观测全部这些天气现象。因此,较为科学合理的方式是分别采用针对各个天气进行特化的模型对每个时间段进行迭代。

在针对美国的一些特殊极端天气情况的测试中,提高分辨率可以在冬季捕捉到更多的极端天气,但在夏天的表现和低分辨率模型并无显著差异。特别的,对于热带气旋,无论是否提高分辨率,模型都无法准确地区分两个气旋。这说明不同时间段可以采用不同的分辨率模型。

为了更好的识别气旋,文章分析了CAM5的数据,分析了它对气旋识别不理想的原因,并针对性地设计了新模型。虽然结果看起来依旧不足以令人满意,但一定程度上减小了误差。这要求整个模型应当能够较为方便地分析每个子模型的数据,为设计新模型提供参考。

总的来说,CAM作为CESM中比较有代表性的一个模块,它的实验流程中遇到的需求对本文设计耦合器有较大的指导意义。

\section{MCT论文阅读报告}

论文1:

M × N COMMUNICATION AND PARALLEL INTERPOLATION IN COMMUNITY CLIMATE SYSTEM MODEL VERSION 3 USING THE MODEL COUPLING TOOLKIT Robert Jacob Jay Larson Everest Ong

论文2:

THE MODEL COUPLING TOOLKIT: A NEW FORTRAN90 TOOLKIT FOR BUILDING MULTIPHYSICS PARALLEL COUPLED MODELS Jay Larson Robert Jacob Everest Ong

	由于MCT要求同时引用这两篇文章,因此我同时阅读了这两篇。其中第一篇主要从算法层面讲述了MCT,而第二篇主要从接口和代码抽象方面给出了文档。

	MCT设计初衷是服务于CESM的各子模型通信交换。CESM通信交换的最初实现是通过一个单线程耦合器分别与多个模型通信来交换外界参数,但随着CESM分辨率提高,所需的线程数也逐步增加,单线程耦合器将会成为瓶颈。因而多线程与多线程间的通信需求成为必要。

	MCT主要解决的是CESM各模块间的通信问题,其模型可以抽象为A模型的m个进程与B模型的n个进程进行通信,而通信数据在两个模型分辨率相同时为p*q的网格,分辨率不同时为p1*q1,p2*q2的网格且需要将传输的数据进行插值。

	接口方面,MCT直接用同类接口替代CESM中的MPI,以最大化地保持原本代码结构和可移植性。MCT定义了两种主要类型AttributeVector和GSMap,前者为一个二维数组用于存储实数数据,而后者用于记录这些数据分别属于哪些进程。虽然每个子模型实现方式各不相同,但它们的数据都可以通过这两种主要类型进行记录和传递,从而满足了可扩展性。

	在数据传递算法中,首先考虑不需要插值的情况,即两个模型的分辨率和网格完全相同。MCT采用一个Router模块记录了每个网格点在另一个模型中属于哪个线程,这个模块可以在两个模型交换GSMap之后各自并行建立,并在初始化后不再改变。Router建立后即可开始进程并行的传输。为了减小通信碎片,在每个发送线程的一对多传输中,所有消息一起发送,并在接收端拆分。

	接下来考虑需要插值的情况。插值可以通过一个矩阵乘法实现,而由于实际模型中网格内有大量0值点,MCT提供了稀疏矩阵存储和乘法来提高插值效率。对稀疏矩阵进行并行乘法时,MCT提供了按矩阵行(源)和矩阵列(目标)拆分并行两种计算方式。在传输时,前者需要在传输前进行插值计算再利用同网格的数据传输算法发送插值后的数据,而后者应当把数据传输到目标模型后再插值计算。由于网格数据格式和插值所需格式不同,MCT提供了Rearrange模块用于改变数据格式以提高插值效率。

	文章进行了各个模块参数的实验和比较,并得出了在当时CESM模型下的一组最优解。从耦合器的设计角度考虑,本文需要更换各种子模型进行实验,而对每个模型应当有不同的最优效率参数。因此,本文将配置格式化为xml文件并利用耦合器自动生成代码,这样可以针对每一组新模型较为方便地测试出一组较优的配置并运行求解。

\section{总结}

开题阶段,在导师和学长的指导和帮助下,我阅读了上述论文,初步了解了实验背景和基准算法,并对研究方向和优化目标有了初步的认识。
