\thusetup{
  %******************************
  % 注意:
  %   1. 配置里面不要出现空行
  %   2. 不需要的配置信息可以删除
  %******************************
  %
  %=====
  % 秘级
  %=====
  secretlevel={秘密},
  secretyear={10},
  %
  %=========
  % 中文信息
  %=========
  ctitle={基于描述的自动耦合实验实现与验证},
  cdegree={工学学士},
  cdepartment={计算机科学与技术系},
  cmajor={计算机科学与技术},
  cauthor={何琦},
  csupervisor={黄震春副研究员},
  % 日期自动使用当前时间,若需指定按如下方式修改:
  % cdate={超新星纪元},
  %
  %=========
  % 英文信息
  %=========
  etitle={Implementation and verification of Automatic Coupling for Earth System Models based on Description Language},
  % 这块比较复杂,需要分情况讨论:
  % 1. 学术型硕士
  %    edegree:必须为Master of Arts或Master of Science(注意大小写)
  %             “哲学、文学、历史学、法学、教育学、艺术学门类,公共管理学科
  %              填写Master of Arts,其它填写Master of Science”
  %    emajor:“获得一级学科授权的学科填写一级学科名称,其它填写二级学科名称”
  % 2. 专业型硕士
  %    edegree:“填写专业学位英文名称全称”
  %    emajor:“工程硕士填写工程领域,其它专业学位不填写此项”
  % 3. 学术型博士
  %    edegree:Doctor of Philosophy(注意大小写)
  %    emajor:“获得一级学科授权的学科填写一级学科名称,其它填写二级学科名称”
  % 4. 专业型博士
  %    edegree:“填写专业学位英文名称全称”
  %    emajor:不填写此项
  edegree={Scholar of Engineering},
  emajor={Computer Science and Technology},
  eauthor={He Qi},
  esupervisor={Associate Professor Huang Zhenchun},
  % 日期自动生成,若需指定按如下方式修改:
  % edate={December, 2005},
  %
  % 关键词用“英文逗号”分割
}

% 定义中英文摘要和关键字
\begin{cabstract}
  耦合技术是解决气候多模式模拟问题的重要手段。目前已有一种基于顶层驱动的耦合器
  可以对多个模式分量进行全局调度、数据交换、统计分析、自动存储恢复等功能。我们
  通过将已存在的模式分量适配进该耦合器来复现模式分量结果的方式,对该耦合器的功能
  进行验证和改进。
  
  我们选用F2000实验作为我们的复现目标。这个实验主要由4个模式分量和1个数据分量构成,
  其原本的工作模式是由驱动模块启动各个模式分量并进行数据交换,然后各个模块分别进行
  各自独立的计算、存储、统计、恢复等功能。现在,我们将这些分量进行重构,在保持其
  原有的计算机能不改变的情况下,将存储、统计、恢复等功能集成到耦合器来处理。

  重构和复现过程验证了耦合器对各种传统模式分量的适配能力。通过从耦合器进行
  的全局数据分析,我们可以将不同模式分量的数据结合在一起。耦合器全局调度的存储和
  恢复功能确保了整个多模式模拟的同步性和稳定性。对某单一模式分量的插拔实验提供了
  模式分量更新无缝对接和等位模式分量数据比较的功能。

  在重构和适配过程中,我们编写了大量调试工具,其中一部分经过标准化后可以帮助耦合器
  进行调试信息动态输出和错误定位。针对模式分量内部可能出现的错误,我们也提供了一套
  嵌入式调试流程。这些调试工具可以在较短的工作量内定位耦合器配置错误或是确认模式
  分量内部错误。

\end{cabstract}

% 如果习惯关键字跟在摘要文字后面,可以用直接命令来设置,如下:
\ckeywords{地球系统模式, 耦合器, 描述语言, 自动耦合, 可行性验证}

\begin{eabstract}
   An abstract of a dissertation is a summary and extraction of research work
   and contributions. Included in an abstract should be description of research
   topic and research objective, brief introduction to methodology and research
   process, and summarization of conclusion and contributions of the
   research. An abstract should be characterized by independence and clarity and
   carry identical information with the dissertation. It should be such that the
   general idea and major contributions of the dissertation are conveyed without
   reading the dissertation.

   An abstract should be concise and to the point. It is a misunderstanding to
   make an abstract an outline of the dissertation and words ``the first
   chapter'', ``the second chapter'' and the like should be avoided in the
   abstract.

   Key words are terms used in a dissertation for indexing, reflecting core
   information of the dissertation. An abstract may contain a maximum of 5 key
   words, with semi-colons used in between to separate one another.
\end{eabstract}

% \ekeywords{\TeX, \LaTeX, CJK, template, thesis}
