\chapter{总结与展望}
\label{cha:further}

在顶层耦合模式分量的开发过程中,我们修改了顶层耦合框架,加入了大量新的适配模块和调试工具;将 CESM 原有的模式分量实现对顶层耦合框架进行了适配和修改,最终在我们自己的耦合代码生成框架下以较高自动化的流程跑通了 F2000 实验,取得了阶段性的成果。

未来,该项目将会增添各种我国科学家自主研发的新型模式分量实现。我们提供的内部库和工具将尽可能为他们的开发提高效率、降低错误率。另外,该项目最终将会在神威太湖之光\footnote{http://www.nsccwx.cn/}超级计算中心进行运行,我们进行的整个代码框架封装和隔离工作为下一步可能的移植铺平了道路。在可预见的未来,该项国家重点研发计划将会成为一套从学科实验到实际使用的高度自动化流程,可以将实验室中得出的新模型在最短时间内投入运算以达成国家战略目标,而本文所做的就是这自动化流程中的微小而又关键的步骤。