\chapter{总结与展望}
\label{cha:further}

在顶层耦合模式分量的开发过程中,本文修改了顶层耦合框架,加入了大量新的适配模块和调试工具;将 CESM 原有的模式分量实现对顶层耦合框架进行了适配和修改,最终在本文自己的耦合代码生成框架下以较高自动化的流程跑通了 F2000 实验,取得了阶段性的成果。

在整个实验过程中,本文学习了来自国外学者和社区开发人员编写的各个类型耦合器与耦合平台的工作原理、大体结构、运行方式、硬件环境等各个方面的基础知识。虽然短暂的工作时间不足以充分了解地球物理学科的各项知识,但本文依然从中获取了编写我国自主研发的耦合平台所需要的部分知识。与此同时,本文还从复现实验的过程中了解了各种国外已有耦合平台存在的问题,并在本文设计的耦合平台中对其进行了改进以避免类似问题的发生。

在底层代码编写的过程中,本文尽可能对所有现存的函数进行依赖解耦,使其对最底层接口的调用更加透明,从而提高了整个代码的优化空间。同时,这也使得未来在对耦合平台中部分代码的维护涉及的代码范围缩窄,从而达到提高效率和降低工作量的目的。在最终版本中,耦合平台调用的外部依赖全部为正在维护的开源库,有效减少了依赖无法满足或闭源库不再支持等风险,能够一定程度上满足该项目自主研发的要求。

在适配各个现存模式分量的代码过程中,本文有效克服了缺乏地球物理学科人力支持的问题,以维持代码行为不变为前提进行适配,最大程度上避免了在重构和适配过程中可能出现的数据错误,在没有外部支持的情况下顺利完成了各个模式分量实现的适配工作,同时也为地球物理学科的后续开发提供了一套基本无需计算机科学人力支持的接口,使得地球物理学科和计算机学科可以分别进行各自的开发并能较为简单高效地对接。

在调试的过程中,本文将遇到的问题进行了充分总结,并将调试过程中编写的工具进行了结构化修改和归档,使其成为了内嵌在耦合平台内的一套强有力的调试系统,能够自动对绝大部分耦合平台内部发生的错误进行追踪和定位。同时,这套工具也可以对模式分量内部可能产生的错误进行预警,并协助进行在整个耦合实验实例中对特定模式分量的嵌入式调试工作。

未来,该项目将会增添各种我国科学家自主研发的新型模式分量实现。本文提供的内部库和工具将尽可能为他们的开发提高效率、降低错误率。另外,该项目最终将会在神威太湖之光\footnote{http://www.nsccwx.cn/}超级计算中心进行运行,本文进行的整个代码框架封装和隔离工作为下一步可能的移植铺平了道路。在可预见的未来,该项国家重点研发计划将会成为一套从学科实验到实际使用的高度自动化流程,可以将实验室中得出的新模型在最短时间内投入运算以达成国家战略目标,而本文所做的就是这自动化流程中的微小而又关键的步骤。
