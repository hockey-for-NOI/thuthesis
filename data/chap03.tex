\chapter{顶层耦合模块}
\label{cha:coupler}

在前面我们对已有的顶层耦合模块进行了介绍。其在设计上具有了运行整个 F2000 实验实例的全部功能,但在代码层面的具体实现中仍然存在问题,需要在适配过程中对其进行修改以达到能够自动对接各个封装后的耦合模式分量并编译运行的目标。这一章实质上是对整个实验中顶层耦合模块的修改所作的文档,并致力于能够在其他实验实例适配的过程中可以进行参考从而可以在遇到类似的问题的时候降低工作量。以下将这些修改按照顶层耦合模式的代码框架进行分别展开。

\section{代码生成模块}

\subsection{元素注册模块 ir}

由于部分模式分量自带一致性检测发现传输数据数量有误,追溯到配置解析模块发现耦合实验实例配置文件存在重复录入和局部名称冲突等问题。为了防止类似问题再次发生,在元素注册 ir 模块中加入了对各个数据域和跨模式分量的全局名称字典,在解析阶段即可进行全局冲突检测。这一步发生在代码生成之前,可以极大的节约编译运行和调试追溯时间。另外,针对部分很可能出现冲突的数据域,追加了可选用的顶层耦合内置前后缀的方式进行区分,降低了跨模式分量实现冲突可能。这一步操作需要内部库函数进行转译操作,且可能会导致在版本更新过程中带来的数据不一致问题,在最终进行数据分析时也要对外部工具进行修改,因此默认为关闭状态。

\section{内部库和编译模块}

这一部分的代码结构进行了调整,现在所有的 Makefile 进行了统一配置,其配置文件位于 baseCpl/src 目录下的 Makefile.conf 文件中,设置了 C++ 和 fortran 编译器及其主要依赖库的编译选项,包括 ESMF , NetCDF , ParallelIO 等外部库的头文件路径和库文件路径。与该模块相关的所有路径名称都要求设置为基于 ABSDIR 主目录宏的相对路径,这是为了在实验实例生成后路径变更时只需要对该宏进行自动修改即可在对应实例的目录下进行自动编译。各个依赖库的目录采用相对路径的方式对该配置文件进行引用,以确保在绝对路径改变后依然能够不依赖于宏地正确引用。

\subsection{数据定义模块 data\_def}

为了确保全局数据存储 global\_var 可以被多种模式分量实现用各自不同的方式进行访问和修改以提高适配面,降低适配难度,将所有分辨率修正前和修正后的数据结构对 MCT 的再封装进行展开,使得各个模式分量可以直接调用而不必通过顶层耦合模式进行解包,达成代码解耦的目的,同时也避免了多进程之间调用和多个模式分量实现之间可能存在的 race-condition 问题。

部分开关标记变量的副本转换为引用,牺牲少许效率换取原生的一致性。这一部分主要考虑到虽然顶层耦合模式存在多进程隔离的一致性保护,但各个模式分量实现编写过程中可能出现对这些变量的直接读取访问,这部分代码可能不受顶层耦合模式的隔离保护,因而可能会触发不一致的问题。值得一提的是,根据目前的框架和各个耦合模式分量的实现,并不存在将一个模式分量实现可写的数据暴露给另一个模式分量实现的情况,因此该设计仅为未来可能出现的该情况进行预先处理,在当前版本并未得到测试。

对通用宏进行了整合,现在默认字符串长度统一为 SHR\_KIND\_CS 和 SHR\_KIND\_CL 两个宏,这两个值在全局类型表文件 shr\_kind\_mod.F90 中被定义为所有不包含路径文本字符串的最大长度和全部文本字符串的最大长度,默认设为 80 和 256 个字符。在本修改前部分不带路径的文件名在 CESM 中被设定为 16 字符长度,因而出现了字符串溢出和命名冲突等问题。为此我们还插入了一个可选的字符串溢出检测模块,该模块开启时会对运行速度产生一定影响,但考虑到与字符串相关操作本身效率不高且已经被主要多进程运算部分尽可能避免,这部分实际测试中对运行效率的影响可以忽略不计,但仍然推荐在调试通过后的最终运行阶段关闭该模块。另外,我们对 CESM 中补充定义的 SHR\_KIND\_CX 和 SHR\_KIND\_CXX 两个宏进行了保留以方便其他模式分量的适配。这两个值预设为 512 和 4096。


